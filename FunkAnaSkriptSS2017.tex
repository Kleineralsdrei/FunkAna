\documentclass[ngerman]{article}
\usepackage{scrlayer-scrpage}
\usepackage[german]{babel}
\usepackage{amsmath}
\usepackage{amssymb}
\usepackage[utf8]{inputenc}
\usepackage{amsthm}
\usepackage{amsfonts}
\usepackage{amsmath}
\usepackage{centernot}
\usepackage{marvosym}
\usepackage{paralist}
\usepackage{amsthm}

\usepackage[nothm]{thmbox}

\pagestyle{headings}

\newtheorem{definition}[section]{Definition}
\newtheorem{bem}[section]{Bemerkung}
\newtheorem{bsp}[section]{Beispiel}

\ihead{Funktionalanalysis - Vorlesungsmitschrift}

\newcommand{\R}{\mathbb{R}}
\newcommand{\N}{\mathbb{N}}
\newcommand{\C}{\mathbb{C}}
\newcommand{\K}{\mathbb{K}}

\newcommand{\hA}{\mathfrak{A}}
\newcommand{\hL}{\mathcal{L}}
%\renewcommand{\def}[2]{\begin{definition}[#1]#2\end{definition}}


%\newcommand{\limfty}{\underset{n\rightarrow\infty \; }{\lim}}

\begin{document}

\section{Grundlagen}
\textbf{Bekannt aus Analysis I-III}
\begin{enumerate}[-]
\item 
Banachraum: vollständiger normierter Vektorraum (wir schreiben $(X, ||\cdot ||_X$)

\item 
Hilbertraum: vollständiger Skalarproduktvektorraum mit $||\cdot || = \sqrt{(\cdot , \cdot )_X}$
\item
Cauchy-Folge: $(x_n),\,  \forall \epsilon > 0\; \exists n \in \N : \forall m \geq n : ||x_m-x_n||<\epsilon$
\item 
vollständiger metrischer Raum, Topologie.
\end{enumerate}

\begin{definition}[Halbnorm, Seminorm]
Sei $X$ ein $\K-Vektorraum$, wobei $\K = \R$ oder $\K = \C$. 
Für $x,y\in X$, $\lambda \in \K$ ist eine Halbnorm oder Seminorm eine Abbildung
$|||\cdot |||:X \rightarrow \R$, die die folgenden Eigenschaften erfüllt:
\begin{enumerate}[(i)]
\item $|||x|||\geq 0$
\item $|||\lambda x||| = |\lambda|\cdot |||x|||$
\item $|||x+y||| \leq |||x||| + |||y|||$
\end{enumerate}
\end{definition}

Eine Norm efüllt zusätzlich noch die Bedingung, dass sie nur dann verschwindet, wenn das Argument verschwindet.

\begin{bem}
\begin{enumerate}[(a)]
\item 
$N:=\{x\in X: |||x|||=0$ bildet einen Unterraum von $X$.
\item 
$X/N$ ist normierter Raum über $||x+N|| := |||x|||$
\item 
X ist vollständiger seminormierter Raum $\Rightarrow$ $X/N$ ist ein Banachraum 
\end{enumerate}
\end{bem}

\begin{bsp}[wichtige Vektorräume]
Sei $(\Omega,\hA,\mu)$ ein Maßraum
\begin{enumerate}[(a)]
\item $\hL^p(\Omega,\mu) = \{f:\Omega \rightarrow \C$ messbar, $\int_\Omega |f|^p d\mu < \infty \}$ ist ein seminormierter Raum mit $|||f|||_p := (\int_\Omega |f|^p d\mu )^{\frac{1}{p}}$

\end{enumerate}
\end{bsp}


\section*{Etwaige Begriffe}
\begin{enumerate}
\item
\textbf{Hausdorfsch, Hausdorffeigenschaft} -  Joa... Kugeln und so.

\end{enumerate}


\end{document}
