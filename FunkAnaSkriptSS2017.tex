\documentclass[ngerman]{report}
%pdf language settings
\usepackage[german]{babel}
\usepackage[utf8]{inputenc}
\usepackage[T1]{fontenc} %provides a better way for westeuropean characters and such 
\setlength{\parindent}{0cm} %noindent for all paragraphs

%pdf page settings
\usepackage[a4paper,top=30mm,bottom=40mm,inner=25mm,outer=35mm]{geometry} 
\usepackage{scrlayer-scrpage} %Headers and footers
%math environments
\usepackage{amssymb}
\usepackage{amsthm}
\usepackage{amsfonts}
\usepackage{amsmath}
\usepackage[nothm]{thmbox} %Decorate theorem statements
%more features for environments and other stuff
\usepackage{centernot} %nicer looking not
\usepackage{marvosym} %some symbols
\usepackage{paralist} %Adds more function to listing environments
\usepackage{xstring} %If then 

\usepackage{marginnote} %allows to do some nice marinnotes
%Headers and Footnotes
\pagestyle{headings}
\ihead{Funktionalanalysis - Vorlesungsmitschrift}


\newtheoremstyle{myStyle}% style, der die Überschrift nicht fett, sondern kursiv erscheinen lässt
  {\topsep}% measure of space to leave above the theorem. E.g.: 3pt
  {\topsep}% measure of space to leave below the theorem. E.g.: 3pt
  {\normalfont}% name of font to use in the body of the theorem
  {0pt}% measure of space to indent
  {\itshape}% name of head font
  {.}% punctuation between head and body
  { }% space after theorem head; " " = normal interword space
  {\thmname{#1}\thmnumber{ #2}\thmnote{ (#3)}}
  
%mathematical environments
\theoremstyle{plain}%Text innen wird kursiv
\newtheorem{thm}{Satz}[chapter]
\newtheorem{lemma}[thm]{Lemma}
\newtheorem{cor}[thm]{Korollar}

\theoremstyle{definition}%Lässt den Text innerhalb nicht mehr kursiv erscheinen
\newtheorem{definition}[thm]{Definition}
\newtheorem{bsp}[thm]{Beispiel}

\theoremstyle{myStyle}
\newtheorem{bem}[thm]{Bemerkung}

%\renewcommand\thechapter{\Roman{chapter}}

%mathematical Macros
\newcommand{\C}{\mathbb{C}}
\newcommand{\R}{\mathbb{R}}
\newcommand{\pR}{\mathbb{R}_{\geq 0}} % R_0^+%
\newcommand{\Q}{\mathbb{Q}}
\newcommand{\Z}{\mathbb{Z}}
\newcommand{\N}{\mathbb{N}}
\newcommand{\K}{\mathbb{K}}

\newcommand{\hA}{\mathfrak{A}}%Script A , hA ~ hässliches A
\newcommand{\hL}{\mathcal{L}}
\newcommand{\tT}{\mathcal{T}} %topologisches T
\newcommand{\B}{\mathcal{B}} %Raum der beschränkten Funktion
\newcommand{\BS}[1][X,Y]{\mathcal{B}(#1)} %"standardraum" der beschränkten Funktion
%Für (mind.) ein Beweis notwendig:
\newcommand{\dX}{\hat{X}}%Dach X
\newcommand{\dd}{\hat{d}}%Dach d
\newcommand{\olx}{\overline{x}} %OverLine x
\newcommand{\oly}{\overline{y}}

\newcommand{\lL}[2][\Omega,\mu]{\text{L}^{#2}(#1)} %Lebesgue L für Matheumgebgung mit Omega, mu als Standard \lL[Raum,Maß]{Dimension}
\newcommand{\ess}{\textnormal{ess}}
\newcommand{\supp}{\textnormal{supp}}
 
\newcommand{\seminorm}[1]{||| #1 |||}
\newcommand{\norm}[1]{\|#1\|}
\newcommand{\intl}[1]{\int_\Omega #1 d\mu} %"Standard"Lebesgue-Integral über Omega
\newcommand{\df}[1][]{% "daraus folgt", also Implikation
	%\IfEqCase{#1}{
	%	{a}{\Rightarrow}
	%	{o}{\overset{\df}{#2}}
	%} 
	\overset{#1}{\Rightarrow}
}
\newcommand{\aq}{\Leftrightarrow} % "äquivalent zu", also Äquivalenz
\newcommand{\U}[2][1]{U_{#1}(#2)} %Umgebung um #2 mit Abstand #1 \U{r}{a}
\newcommand{\EK}{\U{0}} %Umgebung bzw Einheitskugel um 0
\newcommand{\limes}[1][\infty]{\lim_{n \to #1}}

\newcommand{\inv}[1]{#1^{-1}}
\newcommand{\supT}[1][a]{
	\IfEqCase{#1}{
	{A}{\sup_{x \in \overline{\EK}}} 
	{R}{\sup_{x \in \partial\EK}}
	{a}{\sup_{x \in \EK}}
	}
} %sup_EK für die Operatornorm bzw mit Argumet auch Rand oder Abschluss von EK möglich
\newcommand{\disp}{\displaystyle}
\newcommand{\qmarks}[1]{$"\text{#1}"$}
\newcommand{\afs}{" \text{}}
\newcommand{\TODO}{\text{$\mathbb{TODO}$}}
%%%%%%%%%%%%%%%%%%%%%%%%%%%%%%%%%%%%%%%%%%%%%%%%%%%%%%%%%%%%%%%%%%%%%%%%%%%%%%%%%%%%%%%%%%%%%%%%%%%%
%%%%%%%%%%%%%%%%%%%%%%%%%%%%%%%%%%%%%%%%%%%%%%%%%%%%%%%%%%%%%%%%%%%%%%%%%%%%%%%%%%%%%%%%%%%%%%%%%%%%

\begin{document}
\chapter{Funktionalanalysis}
\section{Grundlagen}

\paragraph{Bekannt aus Analysis I-III}

\begin{enumerate}[-]
	\item Banachraum: vollständiger normierter Vektorraum (wir schreiben $(X,\norm{\cdot }_X$) 
	\item Hilbertraum: vollständiger Skalarproduktvektorraum mit $\norm{\cdot } = \sqrt{(\cdot , \cdot )_X}$.  Wobei $(\cdot , \cdot )$ das Skalarprodukt bezeichnet.
	\item Cauchy-Folge: 
					$(x_n),\,  \forall \varepsilon > 0\; \exists n \in \N : \forall m \geq n : \norm{x_m-x_n}<\varepsilon$
	\item vollständiger metrischer Raum, Topologie.
\end{enumerate}

%Definition der Seminorm
\begin{definition}[Halbnorm, Seminorm]

	Sei $X$ ein $\K-Vektorraum$, wobei $\K = \R$ oder $\K = \C$. 
	Für $x,y\in X$, $\lambda \in \K$ ist eine Halbnorm oder Seminorm eine Abbildung
	$\seminorm{\cdot}:X \rightarrow \R$, die die folgenden Eigenschaften erfüllt:

		\begin{enumerate}[(i)]
			\item $\seminorm{x}\geq 0$
			\item $\seminorm{\lambda x} = |\lambda|\cdot \seminorm{x}$
			\item $\seminorm{x+y} \leq \seminorm{x} + \seminorm{y}$
		\end{enumerate}
\end{definition}

Eine Norm efüllt zusätzlich noch die Bedingung, dass sie nur dann verschwindet, wenn das Argument verschwindet.

%Der Kern einer Seminorm bildet einen Unterraum
\begin{bem}
	\begin{enumerate}[(a)]
		\item $N:=\{x\in X: \seminorm{x}=0\}$ bildet einen Unterraum von $X$.
		\item $X/N$ ist ein normierter Raum über(?) $\norm{x+N} := \seminorm{x}$
		\item X ist ein vollständiger seminormierter Raum $\Rightarrow$ $X/N$ ist ein Banachraum 
	\end{enumerate}
\end{bem}

%Beispiele wichtiger Vektorräume
\begin{bsp}[wichtige Vektorräume]
	Sei $(\Omega,\hA,\mu)$ ein Maßraum
		\begin{enumerate}[(a)]
			\item $p\in[1,\infty)\; \hL^p(\Omega,\mu) = \{f:\Omega \rightarrow \C$ messbar, 
						$\int_\Omega |f|^p d\mu < \infty \}$ ist ein seminormierter Raum mit 
						$\seminorm{f}_p := (\int_\Omega |f|^p d\mu )^{\frac{1}{p}}$.\\
						$L^p(\Omega,\mu)$ ist ein vollständiger normierter Raum ($\nearrow$ Ana III).

			\item $\hL^\infty(\Omega,\mu) := \{f:\Omega \rightarrow \C 
						\text{ messbar und essentiell beschränkt} \}$ ist ebenfalls seminormiert mit 
						$\seminorm{f}_\infty := \underset{x\in\Omega}{\ess \sup} |f(x)|$.\\
						$L^\infty(\Omega,\mu)$ ist ein vollständiger normierter Raum.

			\item $p\in [1,\infty],\, |\cdot|$ sei das Zählmaß auf $\N$ und der Maßraum sei gegeben durch 
						$(\N, P(\N), |\cdot|)$.\\
						$\ell^p := \hL^p(\N, |\cdot|)$ heißt Folgenraum und ist ein normierter unendlichdimensionaler Raum.

			\item $\Omega \subseteq \R$ messbar, $\lambda^n$ Lebesgue-Maß auf $\R^n$.
						$L^p(\Omega) := L^p(\Omega,\lambda^n)$ heißt Lebesgue-Raum.

			\item Sei $(\Omega, \tT)$ ein topologischer Raum. 
						$BC(\Omega) := \{ f: \Omega \rightarrow \C\;|\;f 
						\text{ stetig und beschränkt} \}$ versehen mit der Suprenumsnorm ist ein Banachraum.
	\end{enumerate}

\end{bsp}

%Trivia Fakten
\begin{bem}[diverse Fakten]
	Seien $p,q,r\in [1,\infty)$
	\begin{enumerate}[(a)]
		\item $L^p(\Omega,\mu)$ ist ein Banachraum, $L^2(\Omega,\mu)$ ist ein Hilbertraum mit $(f,g)_2 := \int_\Omega f \overline{g} d\mu$
		
		\item Falls $\mu(\Omega) < \infty,\, p\geq r \Rightarrow\; L^p(\Omega,\mu)\subseteq L^r(\Omega,\mu)$
		
		\item Wenn $p\geq r\Rightarrow\; L^r(\Omega,\mu)\cap L^\infty(\Omega,\mu)\subseteq L^p(\Omega,\mu)$
		
		\item $\frac{1}{p}+\frac{1}{q}=1,\,f\in L^p(\Omega,\mu),\,g\in L^q(\Omega,\mu)\Rightarrow\;fg\in 		 L^1(\Omega,\mu)$ mit $\norm{fg}_1\leq\norm{f}_p\norm{g}_q$ (Hölder-Ungleichung). Dies gilt auch für $p=1, q=\infty$ wobei \underline{hier} $\frac{1}{\infty}:=0$.
		
		\item Sei $\Omega\subseteq \R^n$ ein Gebiet. $C^k_0:=\{f:\Omega \rightarrow \C\,|\, \supp f$ kompakt und $f\in C^k(\Omega,\C)\}$ ist dicht in $L^p(\Omega)\;\forall p\in[1,\infty)$. Dies gilt nicht für $p=\infty$, da $f=\textnormal{const}$ oder $f=sign$ sich nicht durch Funktionen aus $C^k_0$ approximieren lassen.
		
		\item $BC(\Omega)$ ist abgeschlossen in $L^\infty (\Omega)$, aber nicht in $L^p(\Omega)$ für $p<\infty$, dennoch ist $BC(\Omega)$ in beiden Fällen ein Unterraum.
	\end{enumerate}
\end{bem}



\section{Lineare Operatoren}

%Definition linearer Operator
	\begin{definition}[linearer Operator]
		Seien $X,Y\; \K-$Vektorräume. Eine Abbildung $T:X\rightarrow Y$ heißt \textit{linearer Operator} wenn 
		$$T(\lambda x+\mu y) = \lambda T(x) + \mu T(y)\;\forall \lambda,\mu \in \K,\,x,y\in X$$ 
		wir schreiben $Tx$ statt $T(x)$.\\
		Wenn $Y=\K$ dann heißt ein linearer Operator $T:X\rightarrow \K$\textit{ Funktional}.\\
		Wenn $X,Y$ normierte $\K-$Vektorräume sind, heißt ein linearer Operator $T$ \textit{beschränkt}, wenn $T(U_1(0)) \subseteq Y$ beschränkt ist. 
		$(\Leftrightarrow \exists M \in \pR$, so dass $\norm{Tx}_Y \leq M$ $\forall x\in X$ mit $\norm{x}_X < 1$)
	\end{definition}
Aus der Definition erkennt man, dass Bilder beschränkter Mengen $M$ unter einem beschränkten linearen Operator $T$ beschränkt sind. Denn $\exists R>0: M \subseteq  U_R(0)$, sodass $T(M) \subseteq T(U_R(0))=T(R\cdot U_1(0))=R\cdot T(U_1(0))$, und dies ist beschränkt.

	\begin{bsp}
		\begin{enumerate}[a)]
			\item $X = \K^n$, $Y = \K^m$, $\{T: X\to Y: T \text{ linearer Operator}\} = \K^{m \times n}.\; T\in\K^{n \times m}$ ist beschränkt. 
			%(Betrachtung nur für den Fall $\K\in\{\R,\C\}$) 
			Denn: 
			$$\norm{T}_\infty = \max_{1\leq i \leq m} \sum^n_{j=1} |t_{ij}| < \infty,\; t_{ij} \text{ sind die Einträge der Matrix }T.$$ Da auf einem endlichdimensionalen Vektorraum alle Normen äquivalent sind, ist $T$ beschränkt.
			\item $T: \lL{1} \to \K, \; Tf := \intl{f}.$ 
				Es gilt $|Tf| = |\intl{f}| \leq \intl{|f|} = \norm{f}_1.$
				Also $|Tf| < 1 \; \forall f \in \lL{1}: \norm{f}_1 < 1 \df T$ beschränkt
		\end{enumerate}
	\end{bsp}

	\begin{thm}
		Seien $X,Y$ normierte Räume, $T: X\to Y$ ein linearer Operator. Dann sind äquivalent:
			\begin{enumerate}[(i)]
				\item $T$ beschränkt,
				\item $T$ ist lipschitz stetig,
				\item $T$ ist gleichmäßig stetig,
				\item $T$ ist stetig,
				\item $T$ stetig in $0$,
				\item $\exists x \in X: T$ stetig in $x$.
			\end{enumerate}
	\end{thm}

%TODO: Über das Beweiskästchen streiten wir nochmal ;) DS (<- Mein Namenskürzel)

	\begin{proof}
		\begin{itemize}[]
			\item $"(i) \df (ii)":$ 
				Sei $M > 0$, so dass $\norm{Tx}_Y \leq M \; \forall x \in U_1(0)$. Es gilt $T0 = 0$.
				Weiterhin gilt für $x \in X\backslash \{0\}$: 
				$$\norm{Tx}_Y = \norm{\: 2 \: \norm{x}_X \: T\left(\frac{x}{2 \norm{x}_X}\right)}
			  =	2 \: \norm{x}_X\norm{T\underbrace{\left(\frac{x}{2 \norm{x}_X}\right)}_{\in \EK}}_Y
				\leq 2 M \norm{x}_X.$$
				Also gilt $\norm{Tx}_Y \leq 2M \norm{x}_X \; \forall x \in \norm{x}_X$ 
				und daraus folgt die Lipschitz Stetigkeit wegen 
				$$ \norm{Tx_1 - Tx_2} = \norm{ T(x_1 - x_2)} \leq 2 M \norm{x_1 - x_2}_X \; \forall x_1, x_2\in X$$ 
											
			\item $"(ii) \df (iii) \df (iv) \df (v) \df (vi)":$ 
				Der Beweis dieser Implikationskette ist Gegenstand der Grundvorlesungen \footnote{Damit meinen wir stets Sätze, die in Analysis\slash LA I,II oder Höhere Analysis bewiesen wurden.}.
				
			\item $"(vi) \df (v)":$ 
				Sei $x \in X$, so dass $T$ stetig in $x$ ist. Sei $(x_n)$ Nullfolge in $X$
				$$\df \limes (x + x_n) = x \df \limes T(x+x_n) = Tx 
				\overset{\text{stetig in 0}}{\df}\limes T x_n = 0 = T\:0$$ 

			\item $"(v) \df (i)":$ Beweis durch Widerspruch:
				Angenommen $T$ ist unbeschränkt $\df \forall n \in \N \; \exists x_n \in \EK$, so dass
				$\norm{Tx_n}_Y \geq n$ ($\df x_n \not = 0 \; \forall n\in\N$). 
				Dann gilt $\frac{x_n}{n} \overset{n\to\infty}{\longrightarrow} 0$,
				aber $\norm{T\frac{x_n}{n}}_Y = \frac{1}{n} \norm{T x_n}_Y \geq \frac{1}{n} \cdot n = 1$
				Das hieße aber $T$ ist unstetig in 0. 

		\end{itemize}
	\end{proof}

	\begin{bem} 
		\begin{enumerate}[a)]
			\item $\B(X,Y) := \{ T: X\to Y: T \text{ beschränkt}\}$
			\item $\B(X) := \B(X,X)$ beides sind $\K-VR$.
			\item $X' := \B(X,\K)$ topologischer Dualraum von	$X$.
		\end{enumerate}
	\end{bem}					
	

	\begin{bem}
		\begin{enumerate}[a)] \addtocounter{enumi}{2}
			\item $Ker\:T$, $Im\:T$ sind UVR. 
			\item $(i) - (vi)$ äquivalent zu $(vii)$:
				Jede beschränkte Menge wird auf eine beschränkte Menge abgebildet.
			\item Es gibt beschränkte lineare Operatoren, so dass $Im\: T$ nicht abgeschlossen $\nearrow$ Übung
			\item $Ker\; T$ abgeschlossen $\forall \; T\in \B(X,Y)$, da $T$ stetig und $Ker\:T = \inv{T}(\{0\})$, wobei $\{0\}$ abgeschlossen in $Y$.
		\end{enumerate}
	\end{bem}						
%%%%%%%%%%%	Satz 1.10
	\begin{thm}[Operatornormen]
		$X,Y$ normierte Räume. $\BS$ normierter Raum mit folgendener Norm
		 $\norm{T} := \disp \sup_{x \in \EK}\norm{Tx}_Y$.
	\end{thm}
	\begin{proof}
		\begin{itemize}[]
				\item (\textit{Positivität:}) 
					$\norm{0} = 0$. Sei $\norm{T} = 0 \df Tx = 0 \; \forall \; x\in\EK$.
					Sei $x\in X$ beliebig. $\df Tx = 2\norm{x}_X \; T \left(\frac{x}{2\norm{x}_X}\right) = 0$
					$\df T = 0$. 

			\item (\textit{Homogenität:}) Sei $\lambda \in \K$, $T\in \BS$. 
				Dann $\norm{\lambda T} = \sup_{x \in \EK} \norm{ (\lambda T) x}_Y
				= |\lambda| \sup_{x \in \EK} \norm{Tx} = |\lambda| \norm{ T}.$
			
			\item (\textit{Dreiecksungleichug:}) Seien $T_1, T_2 \in\BS$. Dann 
				$\disp \norm{T_1 + T_2} = \sup_{x \in \EK}(\norm{T_1x + T_2x}_Y)
				\leq \sup_{x \in \EK}(\norm{T_1x}_Y + \norm{T_2x}_Y)
				\leq \sup_{x_1,x_2  \in \EK}(\norm{T_1x_1}_Y + \norm{T_2x_2}_Y)
				\leq \sup_{x_1 \in \EK}\norm{T_1x_1}_Y + \sup_{x_2 \in \EK}\norm{T_1x_2}_Y
				= \norm{T_1} + \norm{T_2}$
		\end{itemize}
	\end{proof}
		
	\begin{bem}
		Es gilt 
		$\disp \norm{T} = \supT[A]\norm{Tx}_Y = \supT[R] \norm{Tx}_Y
		= \sup_{\overset{x \in X}{x \not = 0}} \frac{\norm{Tx}_Y}{\norm{x}_X}$
		($\nearrow$ Übung).
	\end{bem}
%%%%%%%%%%% Satz 1.12
	\begin{thm}
		$X$ normierter Raum, $Y$ Banachraum. Dann ist $\BS$ Banachraum.
	\end{thm}
		\begin{proof}
			Sei $(T_n)$ $CF$ in $\BS$, d.h. 
			$\forall \; \varepsilon > 0 \exists \; N\in \N \; \forall n,m > N: \norm{T_n - T_m} < \varepsilon$.		
			Also $\norm{T_nx - T_mx}_Y \leq \norm{T_n - T_m} \cdot \norm{x} < \varepsilon \cdot \norm{x} \; \forall x\in X$. 
			Daraus folgt wegen der Vollständigkeit von Y, dass $(T_nx)$ in $Y$ für alle $x\in X$ konvergiert.
			Wir setzen den Grenzwert auf $\disp T: X\to Y,\; Tx := \limes T_nx$. Die so definierte Abbildung, also dieser Grenzwert, erfüllt folgende Eigenschaften: 
				\begin{enumerate}[a)]
					\item $T$ ist ein linearer Operator.
					\item $T$ ist beschränkt.
					\item $\disp \limes \norm{T - T_n} = 0$ (also Normkonvergenz bzw. gleichmäßige Konvergenz)
				\end{enumerate}
				\begin{itemize}[]
					\item \underline{Zu a):} 
						$\disp T(\lambda x_1 + \mu x_2) = \limes T_n(\lambda x_1 + \mu x_2) 
						= \limes (\lambda T_n x_1 + \mu T_n x_2) = \lambda \limes T_nx_1 + \mu \limes T_n x_2
						= \lambda T x_1 + \mu T x_2$
					\item \underline{zu b):}
						Wegen $\norm{T_n - T_m} \geq (\norm{T_n} - \norm{T_m})$ gilt $\norm{T_n}$ ist $CF$ in $\R$
						, also beschränkt: $\disp M := \sup_{n\in\N} \norm{T_n} < \infty$. 
						Für $x\in\EK$ gilt $\disp \norm{Tx}_Y = \limes \norm{T_nx}_Y 
						\leq \limes \norm{T_n} \cdot \norm{x}_X \leq M\cdot \norm{x}_X \leq M$. (vgl. Def 1.5, $"\aq"$)
					\item \underline{zu c):}
						Sei $\varepsilon > 0 \df \exists N\in\N \; \forall m,n > N: \norm{T_n - T_m} < \frac{\varepsilon}{2}.$
						Für $x\in \EK$ gilt somit %(wegen $\disp m \to \infty$) 
						$$\disp \norm{(T-T_n)x} =\lim_{m\to \infty} \norm{(T_m - T_n)x} \leq \frac{\varepsilon}{2}
						\disp \df \norm{T - T_n} = \supT \norm{(T-T_n)x} \leq \frac{\varepsilon}{2} < \varepsilon \; \forall n \geq N$$
						
				\end{itemize}
				Also ist $T\in \BS$ und aufgrund der Beliebigkeit der $CF$, folgt die Vollständigkeit.
		\end{proof}

%%%%%%% Korollar 1.13
	\begin{cor}
		$X$ normierter Raum $\df$ $X'$ Banachraum.
	\end{cor}
%%%%%%% Bemerkung 1.14
	\begin{bem}
		\begin{enumerate}[a)]
			\item $T\in \BS$, $S \in \B(Y,Z) \df ST \in \B(X,Z)$ und 
				$\norm{ST} \leq \norm{S} \cdot \norm{T}$ 
				(gilt wegen $\norm{S(Tx)}_Z \leq \norm{S} \cdot \norm{Tx}_Y
				\leq \norm{S} \cdot \norm{T} \cdot \norm{x}_X \leq M \norm{x}_X$ $\forall x\in X$ und der Linearität von $ST$.) 
			\item $id\in \B(X,X)$, $\norm{id} = 1$.
			\item Aus punktweise Konvergenz $T_nx \to Tx$ folgt
			i.A. \underline{nicht} $\disp \limes T_n = T$ (d.h. $\limes \norm{T_n - T} = 0$).
				\begin{description} \item[Bsp:] 
					$X = \ell^p, p\in [1,\infty)$, $T_n:\ell^p \to \ell^p, \; T_n(x_k) = (x_1,\dots,x_n,0,0,\dots)$ 
					wobei $(x_k) = (x_1,\dots,x_n,\dots).$ Man kann zeigen, dass $T_n \in \B(x)$ $\forall n\in\N$ 
					($\nearrow$ Übung).\par
					Sei $(x_k)\in \ell^p$, $\forall \epsilon > 0 \; \exists N\in\N: (\sum_{k=N+1}^\infty |x_k|^p)^{1\backslash p} < \epsilon.$\ $\norm{T_n(x_k) - x_n}_X = (\sum_{k=N+1}^\infty |x_k|^p)^{1\backslash p} \; \forall n\geq N$.
					Also $\forall x \in X$ $\norm{T_n - x}_X \to 0 \; (n\to \infty).$ Frage: $\norm{T_n - T}_X \to 0$ ?
					Nein! Sei $(x_k^n) = (0,\dots,0,\overset{\text{n+1 Stelle}}{1},0,\dots)$,  
					$\norm{T_n(x_k^n) - x}_X = \norm{(0,\dots,0,-1,0,\dots)})_Y = 1$ 
					$\disp \norm{T_n - T} \overset{Def}{=} \supT \norm{(T_n - T)x}_X \geq 
					\norm{(T_n - T)(\frac{1}{2} (x_k^n)} =\frac{1}{2}\cdot 1$ ($T = idx$) $\forall n\in\N$ 
					$\df \norm{T_n - T} \not\to 0\; (n\to\infty)$
				\end{description}

		\item $T\in \BS$ und $T$ bijektiv. Dann ist $\inv{T}$ i.A. \underline{nicht beschränkt}.
			\begin{description} 
				\item[Bsp.]	$X\in C[0,1], Y= \{f\in C^1([0,1]): f(0) = 0\}$ mit $\disp \norm{x}_X = \sup_{t\in[0,1]}|x(t)|$ und $\norm{\cdot}_X = \norm{\cdot}_Y$ und $T: X\to Y,\; (Tx)(t) = \int_0^t x(s)ds$.
				\begin{itemize}
					\item $\inv{T}=S: Y\to X, Sy = y'$. (Zeige $ST = id_x$ und $TS = id_Y$)
					\item $\inv{T}\not\in\B(Y,X)$ (Sei $y_n(t) = t^n\in Y$, $(\inv{T}y_n)(t) = n\cdot t^{n-1}$
					$\df \norm{y_n}_Y =1 \; \forall n\in \N$, $\norm{\inv{T}y}_X =n \; \forall n\in\N \df \inv{T}$ kann nicht beschränkt sein. 
					($\norm{\inv{T}\frac{1}{2} y_n}_X = \frac{1}{2} \cdot n$ mit $\norm{\frac{1}{2} y_n} = \frac{1}{2})$
				\end{itemize}
				Bem: $Y$ ist nicht vollständig.
			\end{description}
		\end{enumerate}
	\end{bem}

	\begin{thm}
		Sei $X,Y$ normierte $\K-VR$, $T\in \BS$. Dann sind äquivalent:
			\begin{enumerate}[(i)]
				\item $T$ ist injektiv und $\inv{T} \in\B(im(T), X)$ normierter UVR von $Y$.
				\item $\exists \: m > 0: \norm{Tx}_Y \geq m\norm{x}_X \; \forall x\in X$.
			\end{enumerate}
	\end{thm}
	\begin{proof}
		\begin{itemize}[]
			\item $"(i)\df (ii)"$: $\exists \: M>0, \norm{\inv{T}y} \leq M\norm{y} \; \forall y\in imT.$
				Sei $x\in X$ $\exists y\in imT: x = \inv{T}y \df \norm{x}_Y \leq M \norm{Tx}_Y 
				\df \norm{Tx}_Y \geq \frac{1}{M} \norm{x}_X = m\norm{x}_X$
			\item $"(ii) \df (i)"$: Sei $x\in X: Tx = 0$.
				Aus $\norm{Tx} \geq m \norm{x}$ folgt $x = 0$ und damit ist $T injektiv$.
				Sei $y\in imT \; \exists x\in X: Tx = y$ und $\inv{T}y = x $
				$\df[(ii)] \norm{\inv{T}y} = \norm{x} \leq \frac{1}{m} \norm{Tx}_Y = \frac{1}{m} \norm{y}_Y,$
				also $\exists\: M = \frac{1}{m}$, $\norm{\inv{T}y}_X \leq M\norm{y}_Y \; \forall v\in imT$
				$\df \inv{T} \in \B(imT,X)$
		\end{itemize}
	\end{proof}
Die Negation dieser Aussage halten wir explizit fest mit folgendem 
	\begin{cor}
		$T \in \BS$ ($X,Y$ normierte $\K-VR$. Dann sind äquivalent:
			\begin{enumerate}[(i)]
				\item $T$ besitzt \underline{keine} stetige Inverser 
					$\inv{T} : imT\to X.$
				\item $\exists$ Folge $(x_n)$ in $X$, so dass $\norm{x_n} = 1 \; \forall n\in \N$
					und $\disp \limes \norm{T x_n} = 0$
			\end{enumerate}
	\end{cor}

	\begin{definition}
		$X-\K-VR$ mit Norm $\norm{\cdot}_1,\norm{\cdot}_2$. Dann heißt $\norm{\cdot}_1$ 
			\begin{enumerate}[(a)]
				\item \qmarks{stärker} als $\norm{\cdot}_2$, falls gilt
					$\disp \limes \norm{x_n - x}_1 = 0 \df \limes \norm{x_n - x}_2$
				\item \qmarks{schwächer} als $\norm{\cdot}_2$, falls $\norm{\cdot}_2$ stärker ist als $\norm{\cdot}_1$.
				\item \qmarks{äquivalent} falls $\norm{\cdot}_1$ stärker und schwächer ist als $\norm{\cdot}_2$
			\end{enumerate}
	\end{definition}

	\begin{thm}
		$X$ $\K-VR$ mit Norm $\norm{\cdot}_1$,$\norm{\cdot}_2$. Dann gilt 
			\begin{enumerate}[(a)]
				\item $\norm{\cdot}_1$ ist stärker als $\norm{\cdot}_2$ 
					$\aq \exists \: M > 0: \norm{x}_2 \leq M \norm{x}_1 \; \forall x\in X$
				\item $\norm{\cdot}_1$ ist schwächer als $\norm{\cdot}_2$ 
					$\aq \exists \: M > 0: \norm{x}_1 \leq M \norm{x}_2 \; \forall x\in X$
				\item $\norm{\cdot}_1$ ist äquivalent zu $\norm{\cdot}_2$ 
					$\aq \exists \: m,M > 0: m\norm{x}_1 \leq \norm{x}_2 \leq M \norm{x}_1 \; \forall x\in X$
			\end{enumerate}
	\end{thm}
	\begin{proof}
		\begin{enumerate}[zu (a):]
	 		\item \qmarks{$\df$} $id : (X,\norm{\cdot}_1) \to (X,\norm{\cdot}_2)$ ist stetig wegen Vor.
				$\df[S.1.15]$ und weil $id$ linear, $id$ beschränkt, 
				$id\in \B((X,\norm{\cdot}_1), (X,\norm{\cdot}_2)$ d.h. 
				$\exists M > 0: \norm{id(X)}_2 \leq M \norm{x}_1 \; \forall x\in X$.\par
			\qmarks{$\Leftarrow$}	Wissen $\exists M>0: \norm{x}_2 \leq M\norm{x}_1 \; \forall x\in X$.
			Sei $\norm{x_n - x}_1 \to 0 \df \norm{x_n - x}_2 \leq M\norm{x_n - x}_1 \to 0 \; (n\to\infty)$
			$\df \norm{\cdot}_1$ stärker als $\norm{\cdot}_2$.
	 	\end{enumerate}
	\end{proof}

	\begin{definition}
		\marginpar{\scriptsize (sonst auch Homöomorphismus)?}.
		Zwei normierte $\K-VR$ $X,Y$ heißen \qmarks{topologisch isomorph}, falls es ein Isomorphismus 
		$T: X\to Y$ mit $T\in \BS$ und $\inv{T}\in\B(Y,X)$. Dann heißt $T$ topologischer Isomorphismus,
	\end{definition}

	\begin{thm}
		$X, Y$ topologisch isomorph $\aq$ $\exists m,M > 0: T\in \BS$ und injektiv
		$: m\norm{x}_X \leq \norm{Tx}_Y \leq M \norm{x}_X \; \forall x\in X$
	\end{thm}
	\begin{proof}
		'Klar' wegen Satz 1.17 und Satz 1.15.
	\end{proof}
	
	\begin{bem}
		\begin{enumerate}
			\item Falls, $m=M=1$, dann nenn wir $T$ \qmarks{Isometrie}.
			\item Falls $\dim X = \dim Y = n\in \N$: $X,Y$ topologisch isomorph und topologischer Isomorphismus = lineare Bijektion.
		\end{enumerate}
	\end{bem}
%% Satz 1.22
	\begin{thm}[Fortsetzung von stetigen Operatoren]
		$X,Y$ normierte $\K-VR$, $Y$ ein Banachraum, $Z\subset X$, $Z$ dichter UVR.
		$T\in \B(Z,Y)$. Dann existiert ein eindeutiger Operator $\tilde{T} \in \BS$, so dass
		$T|_Z = T$.
	\end{thm}
	\begin{proof}
					TODO: Beweis tippen.
	\end{proof}
%% Satz 1.23
	\begin{thm}
					Ist $T$ normerhaltend (in $\R^n$ die unitären Matrizen $\norm{Tx} = \norm{x}$), so ist $\tilde{T}$ ebenfalls normerhaltend.
	\end{thm}
		\begin{proof}
						TODO: Kurze Begründung. Eigentlich Korollar?
		\end{proof}

	\begin{bsp}[Konstruktion eines unbeschränkten Funktionals]
		Sei $X= \ell^1$ ({Raum der absolut konvergenten Folgen})\par
		Betrachte: $x_0 = (1, \frac{1}{4}, \frac{1}{9},\dots) \in \ell^1$, 
		$\norm{x_0} = \sum_{n=1}^\infty |\frac{1}{n^2}| = \frac{\pi^2}{6} $,\par
		Einheitsvektor $e_k = (\delta_{nk})_{n\in\N}$.\par 
		$\nearrow$ Erzeugnis: \underline{endliche} linear Kombination der Einheitsvektoren 
		$\df \text{span} \{e_k\}_{k_\N} = \{(x_1,x_2,\dots,0,\dots)\}$ (Folgen, die irgendwann zu $0$ werden.)\par
		Die Familie $B := (x_0,e_1,e_2,e_3,\dots)$ ist linear unabhängig.
		$\df B_i$ lässt sich zu Basis $B = (b_i)_{i\in I}$ mit $\N_0 \subset I$ und $b_0 = x_0, b_i = e_i \; \forall i\in \N$ erweitern (überabzählbar).\par
		Sei $x\in X= \ell^1 \df \exists$ eindeutige Darstellung 
		$x = \alpha_0 x_0 +\sum_{\overset{n\in\N}{endlich}} \alpha_n e_n + \sum_{\overset{i\in I\backslash N_0}{endlich}}\alpha_i b_i$. \par
		Definiere das Funktional: $f: \ell^1 \to \K \N?$ mit $x = \alpha_0 x_0 +\sum_{\overset{n\in\N}{endlich}} \alpha_n e_n + \sum_{\overset{i\in I\backslash N_0}{endlich}}\alpha_i b_i \mapsto \alpha_0$\par
	Wir zeigen: $Ker f$ nicht abeschlossen.\par
		Betrachte Folge $(x_n)_{n\in\N}$ mit $x_n \sum_{k=1}^n \frac{1}{k^2}$		 
		$\df x_n\in Ker f \; \forall n\in\N$, da $x_n\in span\{e_k\}_{k\in\N}$. 
		Es gilt jedoch $x_n \to x_0 \not\in Ker f$, da $f(x_0) = 1$.
	\end{bsp}

Nun versuchen wir mit Erfolgt einer waghalsige Verallgemeinerung der geometrischen Reihe im Reellen für Operatoren und Banachräume. $\sum_{k=0}^\infty q^k = \frac{1}{1-q} \; \forall q\in\C$ mit $norm{q} < 1$
%% Satz 1.25 Neumansche Reihe
	\begin{thm}[Neumanansche Reihe] 
		$X$ Banachraum. Sei $T\in \B(X)$. Dann sind äquivalent: 
			\begin{enumerate}[i)]
				\item Die Reihe $\disp \sum_{i=0}^\infty T^k = I_X + T^1 + T^2 + \dots$ ist konvergent bzgl. der Operatornorm.
				\item $\disp \limes \norm{T^n} = 0$
				\item $\exists N\in \N: \norm{T^N} < 1$
				\item $\disp \limes \sup \sqrt[n]{\norm{T^n}} < 1$. 
			\end{enumerate}
		In diesem Fall besitzt $(I-T)$ eine beschränkte Inverse.
		Dies erfüllt $\inv{(I-T)} = \sum_{k=0}^\infty T^k$.
		\marginpar{\scriptsize wenn $\norm{T} \leq 1$ haben wir gewonnen, aber $\norm{T}$ kann groß sein (nilpotente Matrizen)}
	\end{thm}
		\begin{proof}
			\begin{itemize}[]
				\item $\afs i)\df ii) \df iii)"$: \qmarks{klar}
				\item $iii) \df iv)"$: Sei $n\in\N \df \exists \ell\in\N, k\in \{q_0,\dots,N-1\}$, 
					s.d. $n = \ell \cdot N + k$ $\df \ell \leq \frac{n}{N}$
					$\df \norm{T^n} = \norm{(T^n)^\ell T^k} \leq \norm{T^N}^\ell \cdot \norm{T^k}$\par
					Sei $M := \max\{1,\norm{T},\norm{T^2},\dots,\norm{T^{N-1}}\}$
					$\df \norm{T^n} \leq M\norm{T^N}^\ell$ \par
					$\df \sqrt[n]{\norm{T^n}} = \sqrt[n]{\norm{T^N}^\ell}\sqrt[n]{M}$
					$\leq \sqrt[n]{\norm{T^N} \frac{n}{N}} \cdot \sqrt[n]{M}$ 
					$ = \underbrace{\sqrt[N]{\norm{T^N}}}_{< 1} \cdot 
						\underbrace{\sqrt[n]{M}}_{\overset{\to 1}{\text{für } n\to\infty}} 
						\cdot \sqrt[n]{\frac{1}{\norm{T^N}}}$ \par
					$\disp \df \limes \sup \sqrt[n]{\norm{T^n}} < 1$   ($\nearrow$ Wurzelkriterium)
				\item $\afs iv) \df i)\afs$ \TODO
			\end{itemize}
			Noch zu zeigen, wenn $(i) - (iv)$ gilt $\df$ 
				$(I - T) \cdot \sum_{k=0}^\infty T^k = (\sum_{k=0}^\infty T^k) \cdot (I-T) = I$:\par
			Es gilt: $(I-T) \cdot S_n = (I-T) \cdot (\sum_{k=0}^\infty T^k) = \sum_{k=0}^n T^k$
		\end{proof}

	\begin{bem}
		\begin{enumerate}
			\item Wenn $\norm{T} < 1$, dann konvergiert die Neumannsche Reihe.
			\item $\disp \limes\sup\sqrt[n]{\norm{T^n}} <1$ ist nur hinreichend für Invertierbarkeit von $I-T$, wie das Gegenbeispiel $T = 2I$ zeigt.
		\end{enumerate}
	\end{bem}
\begin{bsp}[Fredholmsche Integralgleichung]
Sei $k\in C([a,b]^2)$.
Der Fredholmsche Integraloperator $$K:C([a,b])\to C([a,b]),\; (Kx)(s):=\int^b_a K(s,t)x(t)dt$$ ist stetig, wenn x stetig ist.
Die Fredholmsche Integralgleichung lautet: $$(I-K)x=y,\quad y\in C([a,b]).$$
Und es gilt: $\displaystyle \|Kx\|_\infty \leq \max_{s\in[a,b]} \int^b_a |K(s,t)|dt\cdot\|x\|_\infty$.\\
Wenn nun $\displaystyle \max_{s\in[a,b]} \int^b_a |K(s,t)|dt<1$, dann gilt für alle $ y\in C([a,b]):$
Die Fredholmsche Integralgleichung $(I-K)x=y$ hat genau eine Lösung $x\in C([a,b])$. Diese hängt stetig von $y\in C[a,b]$ ab.
\end{bsp}

\section{Metrische und topologische Räume, Satz von Baire}

\begin{bem}[Erinnerung]
\begin{enumerate}[-]
\item $(X,d)$ metrischer Raum mit Metrik $d$.

\item Kompaktheit, Satz von Bolzano-Weierstraß
\end{enumerate}
\end{bem}

\begin{lemma}
Sei $(X,d)$ ein metrischer Raum. Dann gilt die Vierecksungleichung:
$$|d(x,y)-d(x_1,y_1)| \leq d(x,x_1) +d(y,y_1)\quad \forall x,x_1,y,y_1\in X$$
\end{lemma}
\begin{proof}
$d(x_1,y_1) \leq d(x_1,x)+d(x,y_1) \leq d(x_1,x)+d(x,y)+d(y,y_1)$\\$\df d(x_1,y_1)-d(x,y) \leq d(x,x_1)+d(y,y_1)$. Analog: $d(x,y)-d(x_1,y_1) \leq d(x,x_1)+d(y,y_1)$\par
$\df |d(x,y) - d(x_1,y_1)| \leq d(x,x_1) +d(y,y_1)$
\end{proof}

\begin{bem}[Rekapitulieren Sie folgende Begriffe]
$U_r(x)$ Kugel mit Radius $r$,	$\overline{M}$ Abschluss,	$\mathring{M}$ Innere,	$\partial M$ Rand,	Kompakt,	offene Überdeckung.
\end{bem}

\begin{definition} %Abstandserhaltung, Einbettung und Isometrie
Seien $(X,d_X),(Y,d_Y)$ metrische Räume. Eine Abbildung $f:X\to Y$ heißt
\begin{enumerate}[(a)]
\item \textit{abstandserhaltend} falls $d_X(x,y)=d_Y(f(x),f(y))$ (Bsp.: orthogonale Matrix).
\item \textit{Isometrie} falls abstandserhaltend und surjektiv.
\end{enumerate}
Eine \textit{abstandserhaltende}  Abbildung heißt auch \textit{Einbettung}. Eine \textit{Einbettung} heißt \textit{dicht}, falls $f(X)$ dicht in Y ist.\par
Notation: Wir schreiben $X\subset Y$, falls X in Y eingebettet ist.
\end{definition}

\begin{thm}%Der große Vervollständigungssatz
Jeder metrische Raum $(X,d)$ lässt sich in einen bis auf Isometrie eindeutig bestimmten vollständigen metrischen Raum $(\dX,\dd)$ dicht einbetten. $(\dX,\dd)$ heißt Vervollständigung von $(X,d)$.
\end{thm}
\begin{proof}
\begin{enumerate}[(1)]
\item Konstruktion von $\dX$\par
Sei $CF(X)$ die Menge aller Cauchyfolgen in X. Seien $\overline{x}:=(x_n),\;\overline{y}:=(y_n) \in CF(X)$.\par 
Wir betrachten den \glqq Abstand\grqq  $$ d(\overline{x},\overline{y}):=\lim_{n\to\infty} d_X(x_n,y_n),$$ der dank Lemma 1.29 wohldefiniert ist.\par 
Nun betrachten wir die Relation $\sim\, \subseteq CF(X) \times CF(X)$ mit
$$\overline{x} \sim \overline{y} :\Leftrightarrow d(\olx,\oly) = 0$$
$\glqq \sim \grqq$ ist tatsächlich eine Äquivalenzrelation und unterteilt $CF(X)$ in Äquivalenzklassen. Sei $[x]$ die Äquivalenzklasse des Repräsentanten $\olx$ und $\dX$ die Menge aller Äquivalenzklassen.\par 
Für $\olx,\olx'\in [x]\in \dX,\;\oly,\oly'\in [y]\in\dX$ gilt: 
\begin{equation*}
\begin{split}
& 0=d(\olx,\olx') = \lim_{n\to \infty} d_X((x_n),(x_n'))\\
& 0=d(\oly,\oly') = \lim_{n\to \infty} d_X((y_n),(y_n')).
\end{split}
\end{equation*}
Wegen $d_x(x_n,y_n') \leq d_X(x_n',x_n')+d_X(x_n,y_n)+d_X(y_n,y_n')$\\
$d_x(x_n,y_n) \leq d_X(x_n,x_n')+d_X(x_n',y_n')+d_X(y_n',y_n)$ ist
$$\lim_{n\to\infty} d_X(x_n',y_n')\leq \lim_{n\to\infty} d_X(x_n,y_n) \leq \lim_{n\to\infty} d_X(x_n',y_n')\df d(\olx,\oly)=d(\olx',\oly')$$
und wir können 	wohldefinieren: $\dd([x],[y]):=d(\olx,\oly)\df \dd$ ist Metrik auf $\dX$.

\item Konstruktion einer dichten Einbettung $f:X\to \dX$\par
Für $x\in X$ sei $f(x):=[(x,x,x,\dots)]$.\\
Es gilt für $x,y\in X$: $\dd(f(x),f(y)) = \lim_{n\to\infty}d_X(x,y)=d_X(x,y)$.\par 
Wir zeigen nun, dass $f(X)$ dicht in $\dX$ liegt. Sei $[x]\in\dX,\;\olx=(x_n)$, da nun $(x_n)$ eine Cauchyfolge in $X$ ist, ist:
$$\forall \varepsilon >0\; \exists N\in\N : d(x_n,x_m)< \varepsilon \;\;\forall n,m \geq N$$
Wir betrachten nun $\olx_N := (x_N,x_N,x_N,\dots)$ $$\df \dd(f(x_N),[x])=\lim_{n\to\infty} d_X(x_N,x_n)\leq \varepsilon$$ 
Damit ist $f(x_N)\to [x]$ für $\varepsilon \to 0$ (oder $N\to\infty$?).

\item Vollständigkeit von $\dX$\par
Sei $([x]_j)$ eine Cauchyfolge in $\dX$. Zu jedem $[x]_j\in \dX\;\exists y_j \in X$ so dass $\dd([x]_j,f(y_j)<\frac{1}{j}$, da f(X) dicht in $\dX$ ist.
$$\df d_X(y_j,y_k) = \dd(f(y_j),f(y_k))\leq \dd(f(y_j),[x]_j)+\dd([x]_j,[x]_k)+\dd([x]_k,f(y_k))<\frac{1}{j}+\dd([x]_j,[x]_k) + \frac{1}{k}$$
$\df (y_j)$ ist eine Cauchyfolge in $X$, $y := (y_j)\in CF(X) \df\;[y]\in\dX$ ist der Kandidat für den Grenzwert der Cauchyfolge.
$$\dd([x]_j,[y])\leq \dd([x]_j,f(y_j))+\dd(f(y_j),[y])<\frac{1}{j}+\lim_{k\to\infty}d_X(y_j,y_k)\df \lim_{j\to\infty} \dd([x]_j,[y])=0$$
das heißt $[x]_j\to[y]$ für $j\to\infty$

\item Eindeutigkeit von $\dX$ im folgenden Sinne: ist $\tilde{X}$ eine weitere Vervollständigung von $X$, so sind $\dX,\tilde{X}$ isometrisch zueinander.\par 
Sei also $(H,d_H)$ ein vollständiger metrischer Raum mit $X\subseteq H$,
$d_H(x,y)=d_X(x,y)$ $\forall x,y\in X$ und $\overline{X} = H$.\par
Unser Ziel ist es, eine Isometrie $g:\dX\to H$ zu bauen.\\
Sei $[x]\in\dX,$ $\olx=(x_n)\in[x]\in\dX$, da $H$ vollständig ist $\exists h\in H$ so dass $\lim_{n\to\infty}d_H(x_n,h)=0$\par 
Wir betrachten $g:\dX\to H$,  $[x]\mapsto h$ wie oben.\par 
$g$ ist surjektiv, da für $h\in H \df \exists \olx = (x_n)\in CF(X)$ so dass $\lim_{n\to\infty}d_H(x_n,h)=0$, also $g([x])=h$\\
$g$ ist abstandserhaltend, da für $[x],[y]\in \dX$ gilt
$$\dd([x],[y])=\lim_{n\to\infty} d_X(x_n,y_n) = \lim_{n\to\infty} d_H(x_n,y_n)=d_H(g([x]),g([y])).$$
\end{enumerate}
\end{proof}

\begin{definition}%Durchmesser
Sei $(X,d)$ ein metrischer Raum, $M\subseteq X,\,M\not= \emptyset$.
Wir definieren den Durchmesser von M durch
$$\delta(M):= \sup\left\lbrace d(x,y):x,y\in M\right\rbrace.$$
\end{definition}

Der folgende Satz ist eine Verallgemeinerung des Intervallschachtelungsprinzips aus $\R$.
\begin{thm}[Cantorscher Durchschnittssatz]
Sei $(X,d)$ ein metrischer Raum, der vollständig ist. $(F_n)$ eine Folge von abgeschlossen Teilmengen mit $F_n \not = \emptyset\;\, \forall n\in\N$, $F_1\supseteq F_2\supseteq \dots$ und $\displaystyle \lim_{n\to\infty} \delta(F_n)=0$\par 
$$\df \exists! x_0\in X: \underset{n\in\N}{\cap}F_n = \{x_0\}$$
\end{thm}
\begin{proof}
klar!
\end{proof}







%%%%%%%%%%%%%%%%%%%%%%%%%%%%%%%%%%%%%%%%%%%%%%%%%%%%%%%%%%%%%%%%%%%%%%%%%%%%%%%%%%
%Zusätze. Alles was man noch definieren, erklären, beweisen, oder erwähnen möchte.
%%%%%%%%%%%%%%%%%%%%%%%%%%%%%%%%%%%%%%%%%%%%%%%%%%%%%%%%%%%%%%%%%%%%%%%%%%%%%%%%%%

\section*{Etwaige Begriffe}
	\begin{enumerate}
		\item \textbf{Hausdorffsch, Hausdorffeigenschaft} - Eine Menge heißt \textit{hausdorffsch}, wenn je zwei versch. Punkte stets disjunkte Umgebungen haben. Metrische Räume sind zum Beispiel hausdorffsch, da zwei versch. Punkte stets einen Abstand $> 0$ haben. Für ein Gegenbeispiel $\nearrow$ topologischer Raum

		\item \textbf{essentiell beschränkt} - 
					$(\Omega, \hA,\mu)$ sei ein Maßraum. Eine Funktion $f: \Omega \rightarrow \R$ heißt essentiell beschränkt, falls 
					$$\underset{x\in\Omega}{\ess \sup} |f(x)| := {\inf_{\underset{\mu(N)=0}{N \in \hA}} }  
					\sup_{x\in \Omega\backslash N} |f(x)| < \infty$$
					oder auch: f ist fast überall beschränkt. 
					Ein Beispiel ist $f(x) : = x\cdot \chi_\Q(x)$ und $\mu = \lambda$, da $f$ nur auf $\Q$ nicht null ist, und $\Q$ ist Lesbesgue-Nullmenge. 

		\item \textbf{topologischer Raum} $(X,\tT)$ - Sei $X$ eine Menge und $\tT\subseteq P(X)$. Die Elemente von $\tT$ sind die \textit{offenen Mengen}. $\tT$ definiert eine \textit{Topologie}, wenn folgende Eigenschaften erfüllt sind:
\begin{enumerate}[(i)]
\item $\emptyset,\,X\in \tT$

\item $A_i\in\tT$ für $i\in I$, $\N \supset I$ endlich $\df$ $\cap_{i\in I} A_i\in\tT$

\item $A_i\in\tT$ für $i\in I$, $I$ bel. Indexmenge $\df$ $\cup_{i\in I}A_i \in \tT$
\end{enumerate}
$(X,\tT)$ ist der \textit{topologische Raum}.\par
Ein Beispiel, für einen topologischen Raum sind die metrischen Räume $(X,d)$: $d$ induziert dann eine Topologie auf $X$, die offenen Mengen sind nämlich durch $d$ bestimmt.\par
Sei $M:=\{1,2\},\dots$
\begin{enumerate}[]
\item $\tT: = \{\emptyset,M\}$. Die triviale Topologie, nur $\emptyset$ und $M$ sind offen.

\item $\tT:=P(M)$. Die diskrete Topologie, alle Mengen sind offen. Die diskrete Metrik induziert genau diese Topologie.

\item $\tT:=\{\emptyset,\{1\},\{1,2\}\}.$ M ist hier nicht hausdorffsch, denn egal welche Umgebung man um 2 betrachtet, man kann nicht erreichen, dass 1 nicht in der gleichen ist.
\end{enumerate}
			
	\end{enumerate} 
 
\end{document}
%TODO: -Layout. -Eigene Kommentare innerhalb des Skripts? 
