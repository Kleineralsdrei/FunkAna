\documentclass[ngerman]{report}
%pdf language settings
\usepackage[german]{babel}
\usepackage[utf8]{inputenc}
\usepackage[T1]{fontenc} %provides a better way for westeuropean characters and such 
\setlength{\parindent}{0cm} %noindent for all paragraphs

%pdf page settings
\usepackage[a4paper,top=30mm,bottom=40mm,inner=25mm,outer=35mm]{geometry} 
\usepackage{scrlayer-scrpage} %Headers and footers
%math environments
\usepackage{amssymb}
\usepackage{amsthm}
\usepackage{amsfonts}
\usepackage{amsmath}
\usepackage[nothm]{thmbox} %Decorate theorem statements
%more features for environments and other stuff
\usepackage{centernot} %nicer looking not
\usepackage{marvosym} %some symbols
\usepackage{paralist} %Adds more function to listing environments
%Headers and Footnotes
\pagestyle{headings}
\ihead{Funktionalanalysis - Vorlesungsmitschrift}


\newtheoremstyle{myStyle}% style, der die Überschrift nicht fett, sondern kursiv erscheinen lässt
  {\topsep}% measure of space to leave above the theorem. E.g.: 3pt
  {\topsep}% measure of space to leave below the theorem. E.g.: 3pt
  {\normalfont}% name of font to use in the body of the theorem
  {0pt}% measure of space to indent
  {\itshape}% name of head font
  {.}% punctuation between head and body
  { }% space after theorem head; " " = normal interword space
  {\thmname{#1}\thmnumber{ #2}\thmnote{ (#3)}}
  
%mathematical environments
\theoremstyle{definition}%Lässt den Text innerhalb nicht mehr kursiv erscheinen
\newtheorem{definition}[section]{Definition}
\newtheorem{bsp}[section]{Beispiel}

\theoremstyle{myStyle}
\newtheorem{bem}[section]{Bemerkung}

\theoremstyle{plain}%Text innen wird kursiv
\newtheorem{thm}[subsection]{Satz}
\newtheorem{lemma}[subsection]{Lemma}
\newtheorem{cor}[subsection]{Korollar}

%mathematical Macros
\newcommand{\C}{\mathbb{C}}
\newcommand{\R}{\mathbb{R}}
\newcommand{\pR}{\mathbb{R}_{\leq 0}} % R_0^+
\newcommand{\Q}{\mathbb{Q}}
\newcommand{\Z}{\mathbb{Z}}
\newcommand{\N}{\mathbb{N}}
\newcommand{\K}{\mathbb{K}}

\newcommand{\hA}{\mathfrak{A}}%Script A , hA ~ hässliches A
\newcommand{\hL}{\mathcal{L}}
\newcommand{\tT}{\mathcal{T}} %topologisches T
\newcommand{\lL}[2][\Omega,\mu]{\text{L}^{#2}(#1)} %Lebesgue L für Matheumgebgung mit Omega, mu als Standard \lL[Raum,Maß]{Dimension}
\newcommand{\ess}{\textnormal{ess}}
\newcommand{\supp}{\textnormal{supp}}
 
\newcommand{\seminorm}[1]{||| #1 |||}
\newcommand{\norm}[1]{\|#1\|}
\newcommand{\intl}[1]{\int_\Omega #1 d\mu} %"Standard"Lebesgue-Integral über Omega
\newcommand{\df}{\Rightarrow} % "daraus folgt", also Implikation

\newcommand{\aq}{\Leftarrow} % "äquivalent zu", also Äquivalenz
\newcommand{\U}[2][1]{U_{#1}(#2)} %Umgebung um #2 mit Abstand #1 \U{r}{a}
\newcommand{\EK}{\U{0}} %Umgebung bzw Einheitskugel um 0
\newcommand{\limes}[1][\infty]{\lim_{n \to #1}}

\begin{document}
\chapter{Grundlagen}

\paragraph{Bekannt aus Analysis I-III}

\begin{enumerate}[-]
	\item Banachraum: vollständiger normierter Vektorraum (wir schreiben $(X,\norm{\cdot }_X$) 
	\item Hilbertraum: vollständiger Skalarproduktvektorraum mit $\norm{\cdot } = \sqrt{(\cdot , \cdot )_X}$.  Wobei $(\cdot , \cdot )$ das Skalarprodukt bezeichnet.
	\item Cauchy-Folge: 
					$(x_n),\,  \forall \epsilon > 0\; \exists n \in \N : \forall m \geq n : \norm{x_m-x_n}<\epsilon$
	\item vollständiger metrischer Raum, Topologie.
\end{enumerate}

%Definition der Seminorm
\begin{definition}[Halbnorm, Seminorm]

	Sei $X$ ein $\K-Vektorraum$, wobei $\K = \R$ oder $\K = \C$. 
	Für $x,y\in X$, $\lambda \in \K$ ist eine Halbnorm oder Seminorm eine Abbildung
	$\seminorm{\cdot}:X \rightarrow \R$, die die folgenden Eigenschaften erfüllt:

		\begin{enumerate}[(i)]
			\item $\seminorm{x}\geq 0$
			\item $\seminorm{\lambda x} = |\lambda|\cdot \seminorm{x}$
			\item $\seminorm{x+y} \leq \seminorm{x} + \seminorm{y}$
		\end{enumerate}
\end{definition}

Eine Norm efüllt zusätzlich noch die Bedingung, dass sie nur dann verschwindet, wenn das Argument verschwindet.

%Der Kern einer Seminorm bildet einen Unterraum
\begin{bem}
	\begin{enumerate}[(a)]
		\item $N:=\{x\in X: \seminorm{x}=0\}$ bildet einen Unterraum von $X$.
		\item $X/N$ ist ein normierter Raum über(?) $\norm{x+N} := \seminorm{x}$
		\item X ist ein vollständiger seminormierter Raum $\Rightarrow$ $X/N$ ist ein Banachraum 
	\end{enumerate}
\end{bem}

%Beispiele wichtiger Vektorräume
\begin{bsp}[wichtige Vektorräume]
	Sei $(\Omega,\hA,\mu)$ ein Maßraum
		\begin{enumerate}[(a)]
			\item $p\in[1,\infty)\; \hL^p(\Omega,\mu) = \{f:\Omega \rightarrow \C$ messbar, 
						$\int_\Omega |f|^p d\mu < \infty \}$ ist ein seminormierter Raum mit 
						$\seminorm{f}_p := (\int_\Omega |f|^p d\mu )^{\frac{1}{p}}$.\\
						$L^p(\Omega,\mu)$ ist ein vollständiger normierter Raum ($\nearrow$ Ana III).

			\item $\hL^\infty(\Omega,\mu) := \{f:\Omega \rightarrow \C 
						\text{ messbar und essentiell beschränkt} \}$ ist ebenfalls seminormiert mit 
						$\seminorm{f}_\infty := \underset{x\in\Omega}{\ess \sup} |f(x)|$.\\
						$L^\infty(\Omega,\mu)$ ist ein vollständiger normierter Raum.

			\item $p\in [1,\infty],\, |\cdot|$ sei das Zählmaß auf $\N$ und der Maßraum sei gegeben durch 
						$(\N, P(\N), |\cdot|)$.\\
						$\ell^p := \hL^p(\N, |\cdot|)$ heißt Folgenraum und ist ein normierter unendlichdimensionaler Raum.

			\item $\Omega \subseteq \R$ messbar, $\lambda^n$ Lebesgue-Maß auf $\R^n$.
						$L^p(\Omega) := L^p(\Omega,\lambda^n)$ heißt Lebesgue-Raum.

			\item Sei $(\Omega, \tT)$ ein topologischer Raum. 
						$BC(\Omega) := \{ f: \Omega \rightarrow \C\;|\;f 
						\text{ stetig und beschränkt} \}$ versehen mit der Suprenumsnorm ist ein Banachraum.
	\end{enumerate}

\end{bsp}

%Trivia Fakten
\begin{bem}[diverse Fakten]
	Seien $p,q,r\in [1,\infty)$
	\begin{enumerate}[(a)]
		\item $L^p(\Omega,\mu)$ ist ein Banachraum, $L^2(\Omega,\mu)$ ist ein Hilbertraum mit $(f,g)_2 := \int_\Omega f \overline{g} d\mu$
		
		\item Falls $\mu(\Omega) < \infty,\, p\geq r \Rightarrow\; L^p(\Omega,\mu)\subseteq L^r(\Omega,\mu)$
		
		\item Wenn $p\geq r\Rightarrow\; L^p(\Omega,\mu)\cap L^\infty(\Omega,\mu)\subseteq L^p(\Omega,\mu)$
		
		\item $\frac{1}{p}+\frac{1}{q}=1,\,f\in L^p(\Omega,\mu),\,g\in L^q(\Omega,\mu)\Rightarrow\;fg\in 		 L^1(\Omega,\mu)$ mit $\norm{fg}_1\leq\norm{f}_p\norm{g}_q$ (Hölder-Ungleichung). Dies gilt auch für $p=1, q=\infty$ wobei \underline{hier} $\frac{1}{\infty}:=0$.
		
		\item Sei $\Omega\subseteq \R^n$ ein Gebiet. $C^k_0:=\{f:\Omega \rightarrow \C\,|\, \supp f$ kompakt und $f\in C^k(\Omega,\C)\}$ ist dicht in $L^p(\Omega)\;\forall p\in[1,\infty)$. Dies gilt nicht für $p=\infty$, da $f=\textnormal{const}$ oder $f=sign$ sich nicht durch Funktionen aus $C^k_0$ approximieren lassen.
		
		\item $BC(\Omega)$ ist bzgl. Folgenbildung abgeschlossen in $L^\infty (\Omega)$, aber nicht in $L^p(\Omega)$ für $p<\infty$.
	\end{enumerate}
\end{bem}



\chapter{Lineare Operatoren}%Im Skript ist Lineare Operatoren nur I.2 und kein eigenes Kapitel?

%Definition linearer Operator
	\begin{definition}[linearer Operator]
		Seien $X,Y\; \K-$Vektorräume. Eine Abbildung $T:X\rightarrow Y$ heißt \textit{linearer Operator} wenn 
		$$T(\lambda x+\mu y) = \lambda T(x) + \mu T(y)\;\forall \lambda,\mu \in \K,\,x,y\in X$$ 
		wir schreiben $Tx$ statt $T(x)$.\\
		Wenn $Y=\K$ dann heißt ein linearer Operator $T:X\rightarrow \K$\textit{ Funktional}.\\
		Wenn $X,Y$ normierte $\K-$Vektorräume sind, heißt ein linearer Operator $T$ \textit{beschränkt}, wenn $T(U_1(0)) \subseteq Y$ beschränkt ist. 
		$(\Leftrightarrow: \exists M \in \pR$, so dass $\norm{Tx}_Y \subset M$ $\forall x\in X$ mit $\norm{x}_X < 1$)
	\end{definition}

	\begin{bsp}
		\begin{enumerate}[a)]
			\item $X = \K^n$, $Y = \K^m$, $\{T: X\to Y: T \text{ linearer Operator}\} = \K^{n \times n}$
			(beschränkt?)
			\item $T: \lL{1} \to \K, \; Tf = \intl{f}.$ 
				Es gilt $|Tf| = |\intl{f}| \leq \intl{|f|} = \norm{f}_1.$
				Also $|Tf| < 1 \; \forall f \in \lL{1}: \norm{f}_1 < 1 \df T$ beschränkt
		\end{enumerate}
	\end{bsp}

	\begin{thm}
		$X,Y$ normierte Räume. $T: X\to Y$ lineare Operator. Dann sind äquivalent:
			\begin{enumerate}[(i)]
				\item $T$ beschränkt,
				\item $T$ ist lipschitz stetig,
				\item $T$ ist gleichmäßig stetig,
				\item $T$ ist stetig,
				\item $T$ stetig in $0$,
				\item $\exists x \in X: T$ stetig in $x$.
			\end{enumerate}
	\end{thm}

%TODO: Über das Beweiskästchen streiten wir nochmal ;) DS (<- Mein Namenskürzel)

	\begin{proof}
		\begin{itemize}[]
			\item $"(i) \df (ii)":$ 
				Sei $M > 0$, so dass $\norm{Tx}_Y \leq M \; \forall x \in U_1(0)$. Es gilt $T0 = 0$.
				Weiterhin gilt für $x \in X\backslash \{0\}$. 
				$$\norm{Tx}_Y = \norm{\: 2 \: \norm{x}_X \: T\left(\frac{x}{2 \norm{x}_X}\right)}
			  =	2 \: \norm{x}_X\norm{T\underbrace{\left(\frac{x}{2 \norm{x}_X}\right)}_{\in \EK}}_Y
				\leq 2 M \norm{x}_X.$$
				Also gilt $\norm{Tx}_Y \leq 2M \norm{x}_X \; \forall x \in \norm{x}_X$ 
				und daraus folgt die Lipschitz Stetigkeit wegen 
				$$ \norm{Tx_1 - Tx_2} = \norm{ T(x_1 - x_2)} \leq 2 M \norm{x_1 - x_2}_X \; \forall x_1, x_2\in X$$ 
											
			\item $"(ii) \df (iii) \df (iv) \df (v) \df (vi)":$ 
				Der Beweis dieser Implikationskette ist Gegenstand der Grundvorlesungen \footnote{Damit meinen wir stets Sätze, die in Analysis\slash LA I,II oder Höhere Analysis bewiesen wurden.}.
				
			\item $"(vi) \df (v)":$ 
				Sei $x \in X$, so dass $T$ stetig in $x$ ist. Sei $(x_n)$ Nullfolge in $X$
				$$\df \limes (X + X_n) = X \df \limes T(X+X_n) = Tx 
				\df \text{(stetig in 0}:)\limes T x_n = 0 = T\:0$$ 

			\item $"(v) \df (i)":$ Beweis durch Widerspruch:
				Angenommen $T$ ist unbeschränkt $\df \forall n \in \N \; \exists x_n \in \EK$, so dass
				$\norm{Tx_n}_Y \geq n.$ ($\df x_n \not = 0 \; \forall n\in\N$). 
				Dann gilt $\frac{x_n}{n} \overset{n\to\infty}{\longrightarrow} 0$,
				aber $\norm{T\frac{x_n}{n}}_Y = \frac{1}{n} \norm{T x_n}_Y \geq \frac{1}{n} \cdot n = 1$
				Das hieße aber $T$ ist unstetig in 0. 

		\end{itemize}
	\end{proof}

\section*{Etwaige Begriffe}
	\begin{enumerate}
		\item \textbf{Hausdorffsch, Hausdorffeigenschaft} - Eine Menge heißt \textit{hausdorffsch}, wenn je zwei versch. Punkte stets disjunkte Umgebungen haben. Metrische Räume sind zum Beispiel hausdorffsch, da zwei versch. Punkte stets einen Abstand $> 0$ haben.

		\item \textbf{essentiell beschränkt} - 
					$(\Omega, \hA,\mu)$ sei ein Maßraum. Eine Funktion $f: \Omega \rightarrow \R$ heißt essentiell beschränkt, falls 
					$$\underset{x\in\Omega}{\ess \sup} |f(x)| := {\inf_{\underset{\mu(N)=0}{N \in \hA}} }  
					\sup_{x\in \Omega\backslash N} |f(x)| < \infty$$
					oder auch: f ist fast überall beschränkt. 
					Ein Beispiel ist $f(x) : = x\cdot \chi_\Q(x)$ und $\mu = \lambda$, da $f$ nur auf $\Q$ nicht null ist, und $\Q$ ist Lesbesgue-Nullmenge. 

		\item \textbf{topologischer Raum} $(X,\tT)$ - Ein Raum, dessen offene Mengen durch $\tT$ gegeben sind, wobei die offenen Mengen die bekannten Eigenschaften behalten sollen.
	\end{enumerate} 

\end{document}
%TODO: -Layout. -Eigene Kommentare innerhalb des Skripts? -bis zum 07.04 aktualisieren!!!!!!!!!!!!!!!!!!!!!1
